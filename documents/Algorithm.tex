\documentclass[a4paper, 11pt]{article}
\usepackage{graphicx}
\usepackage{amsmath,array}
%\usepackage{tablefootnote}
\usepackage{epstopdf}
\author{Yuxi Chen}
\title{Algorithm}
\begin{document}
\maketitle
\newpage

\section{Unit conversion}

A physical quantity contains two parts: the value times the unit. For example, the weight of a table is $w = 50 kg = 50,000 g$, where 50 and 50,000 are the values, and kg and g are the units. If we define a new unit wg, which is defined as $1 wg = 2.5 kg$, then the table weight is $w = 20 wg$. In general, a physical quantity can be expressed as $u = \bar{u} u^*$, where $\bar{u}$ is the value/number and $u^*$ is the unit.  

\subsection{Mass normalization}

With the CGS unit, the particle motion equation is
\begin{eqnarray}
\frac{dv}{dt} = \frac{q}{m}(E + \frac{v}{c} \times B) 
\label{eq:velocity_update}
\end{eqnarray}
In FLEKS, all quantities are normalized. The normalization should not introduce new constants. So, from eq.\ref{eq:velocity_update}, the following expression is obtaied
\begin{equation}
\frac{q_{cgs}^* E_{cgs}^* t_{cgs}^*}{m_{cgs}^* v_{cgs}^*} = 1.
\label{eq:E1}
\end{equation}
From Gauss's law $\nabla\cdot E = 4\pi \rho$, we obtain 
\begin{equation}
E_{cgs}^* = \rho_{cgs} ^*x_{cgs}^*,
\label{eq:E2}
\end{equation}
where $\rho_{cgs}^* = q_{cgs}^*/x_{cgs}^{*3}$ is the charge density instead of mass density. From eq.\ref{eq:E1} and eq.\ref{eq:E2}, we obtain 
\begin{equation}
q_{cgs}^*=v_{cgs}^* \sqrt{m_{cgs}^* x_{cgs}^*}.
\label{eq:q1}
\end{equation}

In the code, $\bar{q}/(\bar{m}\bar{c}) = 1$ is required for a proton, so the normalization units $q_{cgs}^*$ and $m_{cgs}^*$ must satisfy 
\begin{equation}
q_{cgs}^*/(m_{cgs}^*v_{cgs}^*) = q_{cgs,p}/(m_{cgs,p}c_{cgs}),     
\end{equation}
where $q_{cgs,p}$ and $m_{cgs,p}$ are proton charge and mass in CGS unit, respectively. With eq.\ref{eq:q1}, we obtain the normalization mass in CGS unit:
\begin{eqnarray}
m_{cgs}^* = x_{cgs}^*(\frac{c_{cgs}m_{cgs,p}}{q_{cgs,p}})^2.
\end{eqnarray}
We note that $m_{cgs}^*$ also consists of the \textit{value} and the unit \textit{g}, and the expression above calculates its \textit{value}. Its \textit{value} in SI unit is calculated inside BATSRUS::ModPIC.f90, where the proton charge and mass are known in SI units. We need to convert the expression above into SI unit. Again, $m_{cgs}^*$ and $m_{SI}^*$ are the \textit{values} of the normalization mass in CGS and SI unit, respectively. If the normalization mass is 1.5kg, then $m_{SI}^* = 1.5$ and $m_{SI}^* = 1500$. The conversion is $m_{cgs}^* = 1000m_{SI}^*$. For the charge unit, $1C = 3\times 10^9 esu$ assume the speed of light is $3 \times 10^8m/s$. So, we have the following expression:
\begin{eqnarray}
1000m_{SI}^* = 100*x_{SI}^*(\frac{100*c_{SI}*1000*m_{SI,p}}{3\times10^9 q_{SI,p}})^2 \\
m_{SI}^* = 10^7x_{SI}^*(\frac{m_{SI,p}}{q_{SI,p}})^2.
\end{eqnarray}

Once $m_{SI}^*$ is obtained, it is passed to FLEKS and converted to $m_{cgs}^*$. 

\subsection{Length and velocity normalization}

These two are free parameters. 

\subsection{Charge and current normalization}

$q_{cgs}^*=v_{cgs}^* \sqrt{m_{cgs}^* x_{cgs}^*}$, as it is shown in eq.\ref{eq:q1}. $j_{cgs}^* = q_{cgs}^*v_{cgs}^*/x_{cgs}^{*3}$

\subsection{B and E normalization}

B and E have the same unit in CGS. Substituting eq.\ref{eq:q1} into eq.\ref{eq:E1} to obtain $E_{cgs}^* = B_{cgs}^* = \sqrt{\frac{m_{cgs}^*}{x_{cgs}^*3}}v_{cgs}^*$ 

\subsection{The consequences of changing light speed}

Reducing the light speed changes the EM wave speed in the simulation, but it does change the interaction between magnetic field and particles. When we discuss the mass normalization, $\bar{q}/(\bar{m}\bar{c}) = 1$ instead of $\bar{q}/\bar{m} = 1$ is required, because there is no assumption of $\bar{c}=1$. The light speed in eq.\ref{eq:velocity_update} is always the physical light speed no matter what $v_{cgs}^*$ is choosen. For example, if $v_{cgs}^* = 0.1c$, $\bar{c}$ has to be 10, otherwise, $\bar{q}/(\bar{m}\bar{c}) = 1$ can not be satisfied. 

Since reducing $v_{cgs}^*$ does not change the interaction between magnetic field and particle, the particle gyromotion, including gyro-radius and gyro-frequency, does not change. The inertial length is the gyro-radius of a particle with Alfven velocity, and neither gyromotion nor Alfven velocity changes, so the inertial length does not change with $v_{cgs}^*$.

How about the interaction between the electric field and particles? Since $\bar{c} = \bar{q}/\bar{m}$, which is not necessarily be 1, the Coulomb force for a proton should be $\bar{q}/\bar{m}E = \bar{c}E$. However, in the code, $\bar{c}$ is ignored, which suggests the Coulomb force in the simulation is $\bar{c}$ times weaker than in reality. Is this reasonable? What is its consequence? It is to be clarified. 

Most PIC simulations use reduced light speed, maybe the results are interpreted in a different way than FLEKS/MHD-EPIC, but I believe they also change the ratio between the Coulomb force and the $v \times B$ force. So, our approach of reducing light speed is probably reasonable. 

\section{Particle mover}

\subsection{Boris algorithm}

FLEKS uses Boris particle mover: 
\begin{eqnarray}
\frac{\mathbf{v}^{n+1} - \mathbf{v}^{n}}{\Delta t} = \frac{q}{m}(\mathbf{E}^{n+\theta} + \bar{\mathbf{v}}\times\mathbf{B}) \\
\frac{\mathbf{x}^{n+1/2} - \mathbf{x}^{n-1/2}}{\Delta t} = \mathbf{v}^n
\end{eqnarray}
where $\mathbf{v}$ is defined as $\bar{\mathbf{v}} = \frac{\mathbf{v}^{n+1} + \mathbf{v}^{n}}{2}$. The traditional method updates the velocity with the following steps.

\begin{enumerate}
    \item Acceleration 
\begin{eqnarray}
    \mathbf{v}^- = \mathbf{v}^{n} + \frac{q\Delta t}{2m}\mathbf{E}^{n+\theta}
\end{eqnarray}

    \item Rotation
\begin{eqnarray} 
    \mathbf{a} = \mathbf{v}^- + \mathbf{v}^- \times \mathbf{t} \\
    \mathbf{v}^+ = \mathbf{v}^- + \mathbf{a} \times \mathbf{s}
\end{eqnarray}
    where $\mathbf{t} = \frac{q\Delta t}{2m} \mathbf{B}$, and $\mathbf{s} = \frac{2\mathbf{t}}{1+t^2}$.

    \item  Acceleration
\begin{eqnarray}    
    \mathbf{v}^{n+1} = \mathbf{v}^+ + \frac{q\Delta t}{2m}\mathbf{E}^{n+\theta}
\end{eqnarray}
\end{enumerate}

FLEKS solves the same velocity equation, but the implementation is slightly different.
\begin{enumerate}
    \item Acceleration 
\begin{eqnarray}
    \mathbf{v}^- = \mathbf{v}^{n} + \frac{q\Delta t}{2m}\mathbf{E}^{n+\theta}
\end{eqnarray}

\item Calculate $\bar{\mathbf{v}}$. 

With the expressions of $\mathbf{v}^+$ and $\mathbf{v}^-$ above, it is easy (it is probably not straightforwad, but I do not know how to number equations in markdown) to obtain $\mathbf{v}^{n+1} - \mathbf{v}^{n-1} = \mathbf{v}^+ - \mathbf{v}^- + \frac{q\Delta t}{m} \mathbf{E}^{n+\theta}$. Substitute it into the velocity update equation to remove electric field from the expression $\frac{\mathbf{v}^{+} - \mathbf{v}^{-}}{\Delta t} = \frac{q}{m}\bar{\mathbf{v}}\times\mathbf{B}$. Since $\mathbf{v}^+ = 2\bar{\mathbf{v}} - \mathbf{v}^-$, we obtain the equation for $\bar{\mathbf{v}}$:
\begin{eqnarray}
  \bar{\mathbf{v}} = \mathbf{v}^- + \frac{q\Delta t}{2m} \bar{\mathbf{v}} \times \mathbf{B} = \mathbf{v}^- + \bar{\mathbf{v}} \times \mathbf{t}   
\end{eqnarray}

To solve this equation, we apply $\cdot \mathbf{t}$ to the expression of $\bar{\mathbf{v}}$ to obtain $\bar{\mathbf{v}} \cdot \mathbf{t} = \mathbf{v}^- \cdot \mathbf{t}$. Then we apply $\times \mathbf{t}$ to the expression of $\bar{\mathbf{v}}$ to obtain
\begin{eqnarray}
\bar{\mathbf{v}} \times \mathbf{t} = \mathbf{v}^- \times \mathbf{t} + \bar{\mathbf{v}} \times \mathbf{t} \times \mathbf{t} = \mathbf{v}^- \times \mathbf{t} + (\bar{\mathbf{v}} \cdot \mathbf{t})\mathbf{t} - t^2\bar{\mathbf{v}} = \mathbf{v}^- \times \mathbf{t} + (\mathbf{v}^- \cdot \mathbf{t})\mathbf{t} - t^2\bar{\mathbf{v}}
\end{eqnarray}
Substitute the expression of $\bar{\mathbf{v}} \times \mathbf{t}$ to the expression of $\bar{\mathbf{v}}$ to obtain 
\begin{eqnarray}
\bar{\mathbf{v}} = \frac{\mathbf{v}^- + \mathbf{v}^- \times \mathbf{t} + (\mathbf{v}^- \cdot \mathbf{t})\mathbf{t}}{1+t^2}
\end{eqnarray}

\item Then it is easy to obtain $\mathbf{v}^{n+1} = 2*\bar{\mathbf{v}} - \mathbf{v}^{n}$
\end{enumerate}

\subsection{Relativistic Boris algorithm}

With $\mathbf{u} = \gamma \mathbf{v}$, the velocity update equation is
\begin{eqnarray}
\frac{\mathbf{u}^{n+1} - \mathbf{u}^{n}}{\Delta t} = \frac{q}{m}(\mathbf{E}^{n+\theta} + \bar{\mathbf{u}}\times \frac{\mathbf{B}}{\gamma ^{n+1/2}})
\end{eqnarray}

\begin{enumerate}
    \item Convert $\mathbf{v}$ to $\mathbf{u}$
    \item Aacceleration
\begin{eqnarray}
    \mathbf{u}^- = \mathbf{u}^{n} + \frac{q\Delta t}{2m}\mathbf{E}^{n+\theta}
\end{eqnarray}

\item Calculate $\bar{\mathbf{u}}$ with
\begin{eqnarray}
\bar{\mathbf{u}} = \frac{\mathbf{u}^- + \mathbf{u}^- \times \mathbf{t} + (\mathbf{u}^- \cdot \mathbf{t})\mathbf{t}}{1+t^2}
\end{eqnarray}
where $\mathbf{t} = \frac{q\Delta t}{2m} \mathbf{B}/\gamma^{n+1/2}$. From the definition of $\gamma$ and $\mathbf{u} = \gamma \mathbf{v}$, $\gamma ^{n+1/2} = \sqrt{1+(u^-/c)^2}$ is obtained. 

\item Then it is easy to obtain $\mathbf{u}^{n+1} = 2*\bar{\mathbf{u}} - \mathbf{u}^{n}$
\item  Convert $\mathbf{u}$ back to $\mathbf{v}$
\end{enumerate}

\end{document}